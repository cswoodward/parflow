%=============================================================================
%=============================================================================

\chapter{Getting Started}
\label{Getting Started}

This chapter is an introduction to setting up and running \parflow{}.
In \S~\ref{Installing ParFlow}, we describe how to install \parflow{}.
In \S~\ref{Running the Sample Problem}, we lead the user through a
simple groundwater problem, supplied with the \parflow{} distribution.

%=============================================================================
%=============================================================================

\section{Installing ParFlow}
\label{Installing ParFlow}

\parflow{} requires a Standard ANSI C and Fortran 77 compiler (or f2c)
to build code.  \parflow{} also requires \code{TCL/TK} version 8.0 (or
higher).  \code{TCL/TK} can be obtained from: \\
\code{http://www.sunlabs.com/research/tcl/}.  

The following steps are designed to take you through the steps of
installing \parflow{} from a source distribution.  This assumes that 
a configuration file exists for your system.  If one does not exist
you will need to create a configuration file for your system in
the \code{config} directory of the \parflow{} source tree.

\begin{enumerate}

\item
The following environment variables should be set up in your \file{.profile}
file (if you are using \code{bash} or a bourne syntax shell):

\begin{display}\begin{verbatim}
export PARFLOW_SRC=~/parflow/src
export PARFLOW_DIR=~/parflow/exe
export PARFLOW_HELP=~/parflow/docs
export PARFLOW_HTML_VIEWER=/usr/local/bin/netscape

\end{verbatim}\end{display}

If you are using a \code{csh} like shell you will need the following in your
\file{.cshrc} file:

\begin{display}\begin{verbatim}
setenv PARFLOW_SRC ~/parflow/src
setenv PARFLOW_DIR ~/parflow/exe
setenv PARFLOW_HELP ~/parflow/docs
setenv PARFLOW_HTML_VIEWER /usr/local/bin/netscape
\end{verbatim}\end{display}

The \code{PARFLOW_HTML_VIEWER} variable should be set to a HTML
browser of your choice.  This is used to choose the browser that will
be used to view the online documentation.  The other variables point
to locations of directory structures used by \parflow{}.  The variable
\code{PARFLOW_DIR} specifies the location of the installed version of
\parflow{}  This is where executables and support files will be placed.
If you have a directory which is shared on multiple architectures you
can set different \code{PARFLOW_DIR}s on the different machines (for
example
\code{~/parflow/exe.t3d} and \code{~/parflow/exe.irix}).  \code{PARFLOW_SRC} is the location of the source code for \parflow{} and affiliated tools. \code{PARFLOW_HELP} is the location of the HTML help files.
We will use the \file{~/parflow} directory as the root directory for
building \parflow{} in this user manual; you can use a different
directory if you wish.

%------------------------------
\item

You should also add \code{\$PARFLOW_DIR/bin} to your 
\code{PATH} environment variable.  
If you are using \code{bash} or a bourne syntax shell):

\begin{display}\begin{verbatim}
PATH=$PATH:$PARFLOW_DIR/bin
\end{verbatim}\end{display}

If you are using a \code{csh} like shell:

\begin{display}\begin{verbatim}
set path=($path $PARFLOW_DIR/bin)
\end{verbatim}\end{display}

%------------------------------

\item

Extract the source files from the distribution compressed tar file.
This example assumes the parflow.tar.Z file is in your home directory
and you are building it in a directory ~/parflow.

\begin{display}\begin{verbatim}
mkdir ~/parflow
cd ~/parflow
zcat ../parflow.tar.Z | tar xf -
\end{verbatim}\end{display}

%------------------------------

\item
Next, we need to build the parflow and other tools which make up the
\parflow{} suite.  

\begin{display}\begin{verbatim}
cd $PARFLOW_SRC
./build install 
./build install docs
\end{verbatim}\end{display}

The first command will build \parflow{} and the bundled tools and
install them in the \code{\$PARFLOW_DIR} directory.  The second
command will build and install the documentation.

After the tools are built, be sure to execute the Unix rehash function 
if you are using a \code{csh} like shell.

\begin{display}\begin{verbatim}
rehash
\end{verbatim}\end{display}

%------------------------------

\end{enumerate}

%=============================================================================
%=============================================================================

\section{Running the Sample Problem}
\label{Running the Sample Problem}

Here, we assume that \parflow{} is already built.
The following steps will allow you to run a simple test
problem supplied with the distribution.
\begin{enumerate}

\item
We first create a directory in which to run the problem,
then copy into it some supplied default input files.
So, do the following anywhere in your \file{\$HOME} directory:
\begin{display}\begin{verbatim}
mkdir foo
cd foo
cp $PARFLOW_DIR/test/default_single.pftcl .
chmod 640 *
\end{verbatim}\end{display}
We used the directory name \file{foo} above;
you may use any name you wish.
The last line changes the permissions of the files so that
you may write to them.

\item
Run \parflow{} using the pftcl file as a TCL script
\begin{display}\begin{verbatim}
tclsh default_single.pftcl
\end{verbatim}\end{display}

\end{enumerate}
You have now successfully run a simple \parflow{} problem.
For more information on running \parflow{},
see \S~\ref{Running ParFlow}.

%==============================
\subsection*{Visualizing the Permeability}

Now suppose we want to visualize some of the output.
In particular, let's look at the permeability field generated
by \parflow{} in this run.
Follow these steps:
\begin{enumerate}

\item
Start avs using the \file{pfavs} command.
This will start AVS (the visualization system we are currently
using), load the \parflow{} module library, and set the default
network directory to your installed \parflow{} network directory.

\item
Click on \menu{Network Editor}.
Then from the \menu{AVS Network Editor}, click on \menu{Read Network},
and select file \file{slice3.net}.
This will read in the \file{slice3.net} network which we will
use to look at 3 slice planes of the permeability data.

\item
The panel buttons \menu{read parflow macro} and \menu{read parflow}
should already be selected (highlighted).
From the \menu{File Browser}, select \file{default_single.out.perm.pfb}.
A picture of the permeability field used in the problem should
appear in the \menu{Geometry Viewer} window.

\end{enumerate}
For more information on visualizing \parflow{} output,
see \S~\ref{Visualizing Output with AVS}.

%==============================
\subsection*{Adding a Pumping Well}

Let us change the input problem by adding a pumping well:
\begin{enumerate}

\item
Edit the file \file{default_single.pftcl} using your favorite text editor.

\item 
Add the following lines to the input file near where the existing
well information is in the input file.  You need to replace
the ``Wells.Names'' line with the one included here to get both
wells activated (this value lists the names of the wells):

\begin{display}\begin{verbatim}
pfset Wells.Names {snoopy new_well}

pfset Wells.new_well.InputType                Recirc

pfset Wells.new_well.Cycle		    constant

pfset Wells.new_well.ExtractionType	    Flux
pfset Wells.new_well.InjectionType            Flux

pfset Wells.new_well.X			    10.0 
pfset Wells.new_well.Y			    10.0
pfset Wells.new_well.ExtractionZLower	     5.0
pfset Wells.new_well.ExtractionZUpper	     5.0
pfset Wells.new_well.InjectionZLower	     2.0
pfset Wells.new_well.InjectionZUpper	     2.0

pfset Wells.new_well.ExtractionMethod	    Standard
pfset Wells.new_well.InjectionMethod          Standard

pfset Wells.new_well.alltime.Extraction.Flux.water.Value        	     5.0
pfset Wells.new_well.alltime.Injection.Flux.water.Value		     7.5
pfset Wells.new_well.alltime.Injection.Concentration.water.tce.Fraction 0.1

\end{verbatim}\end{display}

\end{enumerate}
For more information on defining the problem,
see \S~\ref{Defining the Problem}.

%==============================
\subsection*{Converting the Pressure Output}

Next, let us convert the pressure output computed by \parflow{}
to hydraulic head.
Do the following:
\begin{enumerate}

\item
Start the \file{xpftools} utility.
This is a utility which is used to do file conversions and other simple
operations on \parflow{} output files.

\item
To read in the data, use the \menu{Data} pulldown menu and select the
\menu{Load} \menu{ParFlow Binary} option.  

\item
In the file browser select the \file{default_single.out.press.00000.pfb}
file.

\item

Select the \menu{Compute Hydraulic Head} option from the \menu{Functions} 
pulldown menu.

\item 

In the \menu{Compute Hydraulic Head} popup, us the selector error next to
the \menu{Pressure Head} field to select \file{dataset0} (the
\file{default_single.out.press.00000.pfb} file).

\item 

Press the \menu{Compute Function} button at the bottom of the panel.  This
creates a new dataset labeled \file{dataset1}. 

\item
To save the hydraulic head data in a \parflow{} file,
\menu{Close} the \menu{Hydraulic Head} panel and select 
\menu{Save ParFlow Data}
from the \menu{Dataset} pulldown menu.

\item

In the file selector box type the filename
\file{default_single.out.hhead.00000.pfb} 
and save the dataset.  

You have now computed and created a new \parflow{} binary file with the
hydraulic head.

\end{enumerate}
For more information on manipulating and analyzing \parflow{} data,
see \S~\ref{Manipulating Data}.

