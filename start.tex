%=============================================================================
%=============================================================================

\chapter{Getting Started}
\label{Getting Started}

This chapter is an introduction to setting up and running \parflow{}.
In \S~\ref{Installing ParFlow}, we describe how to install \parflow{}.
In \S~\ref{Running the Sample Problem}, we lead the user through a
simple groundwater problem, supplied with the \parflow{} distribution.

%=============================================================================
%=============================================================================

\section{Installing ParFlow}
\label{Installing ParFlow}

\parflow{} now uses a new configure/make system based on
the standard \code{GNU autoconf} configure system.  This replaces the home
grown set of scripts used in previous versions and is much more
portable to a range of machines.

Since \parflow{} is composed of several different parts which might
require different compilers (\emph{e.g.} on some MPP architectures the
front-end machine is different from the nodes) the build process
involves compiling several different parts.

These instructions are for building \parflow{} on a range of serial and parallel linux, unix and OSX machines, including stand-alone single and multi-core to large parallel clusters.  These instructions do \emph{NOT} include compilation on Windows machines.

\parflow{} requires a Standard \emph{ANSI C} and \emph{FORTRAN 90/95} compiler
to build code.  \code{GCC} and \code{gFortran}, available for free on almost every platform, are good options and may be found at:
\begin{center}
\code{http://gcc.gnu.org/}  
\end{center}
and
\begin{center}
\code{http://gcc.gnu.org/wiki/GFortran}
\end{center}
\parflow{} also requires \code{TCL/TK} version 8.0 (or higher).  \code{TCL/TK} can be obtained from:
\begin{center}
\code{http://www.tcl.tk/}
\end{center}
These three packages are often pre-installed on most computers and generally do not need to be installed by the user.
The following steps are designed to take you through the process of
installing \parflow{} from a source distribution.  \parflow{} uses the \code{gnu} package \code{autoconf} to create a configuration file for building and installing the \parflow{} program.  

\begin{enumerate}
\item {\bf Setup} \\
Decide where you wish to install Parflow and associated libraries. The following environment variable should be set up in your \file{.profile} file. Set the environment variable \code{PARFLOW_DIR} to your chosen location (if you are using \code{bash} or a bourne syntax shell):
\begin{display}\begin{verbatim}
export PARFLOW_DIR=~/parflow/exe
\end{verbatim}\end{display}
If you are using a \code{csh} like shell you will need the following in your
\file{.cshrc} file:
\begin{display}\begin{verbatim}
setenv PARFLOW_SRC ~/parflow/src
\end{verbatim}\end{display}

The variable
\code{PARFLOW_DIR} specifies the location of the installed version of
\parflow{}  This is where executables and support files will be placed.
If you have a directory which is shared on multiple architectures you
can set different \code{PARFLOW_DIR}s on the different machines (for
example
\code{~/parflow/exe.t3d} and \code{~/parflow/exe.irix}). 
We will use the \file{~/parflow} directory as the root directory for
building \parflow{} in this user manual; you can use a different
directory if you wish.

You should also add \code{\$PARFLOW_DIR/bin} to your 
\code{PATH} environment variable.  
If you are using \code{bash} or a bourne syntax shell):

\begin{display}\begin{verbatim}
PATH=$PATH:$PARFLOW_DIR/bin
\end{verbatim}\end{display}

If you are using a \code{csh} like shell:

\begin{display}\begin{verbatim}
set path=($path $PARFLOW_DIR/bin)
\end{verbatim}\end{display}

%------------------------------

\item {\bf Extract the source}\\
Extract the source files from the distribution compressed tar file.
This example assumes the \file{parflow.tar.Z} file is in your home directory
and you are building it in a directory \file{~/parflow}.

\begin{display}\begin{verbatim}
mkdir ~/parflow
cd ~/parflow
gunzip ../parflow.tar.Z 
tar -xvf ../parflow.tar
\end{verbatim}\end{display}

%------------------------------

\item {\bf Build and install \parflow{}}\\
This step builds the \parflow{} library and executable.  The library is
used when \parflow{} is used as a component of another simulation
(\emph{e.g.} WRF).  Next, we need to build the parflow and other tools which make up the
\parflow{} suite.  

\begin{display}\begin{verbatim}
cd $PARFLOW_DIR
cd parflow
./configure --prefix=$PARFLOW_DIR --with-amps=mpi1
make 
make install
\end{verbatim}\end{display}

This will build a parallel version of /parflow{} using the MPI1 libraries. You can control build options for /parflow{}, use
\begin{display}\begin{verbatim}
./configure --help 
\end{verbatim}\end{display} to see other configure options.  Note that \parflow{} defaults to building a sequential version so
\code{--with-amps} is needed when building for a parallel computer.  You
can explicitly specify the path to the MPI to use with the
\code{--with-mpi} option to configure.  This controls \emph{AMPS} which stands for \emph{A}nother \emph{M}essage \emph{P}assing \emph{S}ytem.  \emph{AMPS} is a very flexible message-passing layer within \parflow{} that allows a common code core to be quickly and easily adapted to many different parallel environments.

\item {\bf Build and install pftools}\\
\code{pftools} is a package of utilities and a \code{TCL} library that is used to
setup and postprocess Parflow files.  The input files to \parflow{} are
\code{TCL} scripts so \code{TCL} must be installed on the system.

The first command will build \parflow{} and the bundled tools and
install them in the \code{\$PARFLOW_DIR} directory.  The second
command will build and install the documentation.
A typical configure and build looks like:

\begin{display}\begin{verbatim}
cd $PARFLOW_DIR
cd pftools
./configure --prefix=$PARFLOW_DIR --with-amps=mpi1
make 
make install
make doc_install
\end{verbatim}\end{display}

Note that \code{pftools} is \emph{NOT} parallel but some options for how files are
written are based on the communication layer so pftools needs to know
what what used to build the \parflow{} library.

If \code{TCL} is not installed in the system locations (\code{/usr} or \code{/usr/local}) you need
to specify the path with the \code{--with-tcl=<PATH>} configure option.

See \file{./configure --help} for additional configure options for \code{pftools}.

\item {\bf Running a sample problem}\\
There is a test directory that contains not only example scripts of \parflow{} problems but the correct output for these scripts as well.  This may be used to test the compilation process and verify that \parflow{} is installed correctly.  If all went well a sample \parflow{} problem can be run using:

\begin{display}\begin{verbatim}
cd $PARFLOW_DIR
cd test
tclsh default_single.pftcl
\end{verbatim}\end{display}

Note that \code{PAFLOW_DIR} must be set for this to work and it assume tclsh
is in your path.  Make sure to use the same \code{TCL} as was used in the
\code{pftools} configure. The entire suite of test cases may be run at once to test a range of functionality in \parflow{}.  This may be done by:
\begin{display}\begin{verbatim}
cd $PARFLOW_DIR
cd test
make test
\end{verbatim}\end{display}

\item {\bf Notes and other options:}\\
\parflow{} may be compiled with a number of options using the configure script.  Some common options are compiling \code{CLM} as in \cite{MM05,KM08a} to compile with timing and optimization or to use a compiler other than \code{gcc}.  To compile with \code{CLM} add \code{--with-clm} to the configure line such as:
\begin{display}\begin{verbatim}
./configure --prefix=$PARFLOW_DIR --with-amps=mpi1 --with-clm
\end{verbatim}\end{display}

To enable detailed timing of the performance of several different components within \parflow{} use the \code{--enable-timing} option.  To use compiler optimizations use the \code{--enable-opt=STRING} where the \code{=STRING} is an optional flag to specify the level and type of optimization.

It is often desirable to use different C and F90/95 compilers (such as \emph{Intel} or \emph{Porland Group}) to match hardware specifics, for performance reasons or simply personal preference.  To change compilers, set the \code{CC}, \code{FC} and \code{F77} variables (these may include a path too). For example to change to the \emph{Intel} compilers in c-shell:
\begin{display}\begin{verbatim}
setenv CC icc
setenv FC ifort
setenv F77 ifort
\end{verbatim}\end{display}
\end{enumerate}


%% NOTE: we should add stuff about SILO support, Hypre support and other things as needed.

%=============================================================================
%=============================================================================

\section{Running the Sample Problem}
\label{Running the Sample Problem}

Here, we assume that \parflow{} is already built.
The following steps will allow you to run a simple test
problem supplied with the distribution.
\begin{enumerate}

\item
We first create a directory in which to run the problem,
then copy into it some supplied default input files.
So, do the following anywhere in your \file{\$HOME} directory:
\begin{display}\begin{verbatim}
mkdir foo
cd foo
cp $PARFLOW_DIR/test/default_single.tcl .
chmod 640 *
\end{verbatim}\end{display}
We used the directory name \file{foo} above;
you may use any name you wish.
The last line changes the permissions of the files so that
you may write to them.

\item
Run \parflow{} using the pftcl file as a TCL script
\begin{display}\begin{verbatim}
tclsh default_single.tcl
\end{verbatim}\end{display}

\end{enumerate}
You have now successfully run a simple \parflow{} problem.
For more information on running \parflow{},
see \S~\ref{Running ParFlow}.


%==============================
\subsection*{Adding a Pumping Well}

Let us change the input problem by adding a pumping well:
\begin{enumerate}

\item
Edit the file \file{default_single.tcl} using your favorite text editor.

\item 
Add the following lines to the input file near where the existing
well information is in the input file.  You need to replace
the ``Wells.Names'' line with the one included here to get both
wells activated (this value lists the names of the wells):

\begin{display}\begin{verbatim}
pfset Wells.Names {snoopy new_well}

pfset Wells.new_well.InputType                Recirc

pfset Wells.new_well.Cycle		    constant

pfset Wells.new_well.ExtractionType	    Flux
pfset Wells.new_well.InjectionType            Flux

pfset Wells.new_well.X			    10.0 
pfset Wells.new_well.Y			    10.0
pfset Wells.new_well.ExtractionZLower	     5.0
pfset Wells.new_well.ExtractionZUpper	     5.0
pfset Wells.new_well.InjectionZLower	     2.0
pfset Wells.new_well.InjectionZUpper	     2.0

pfset Wells.new_well.ExtractionMethod	    Standard
pfset Wells.new_well.InjectionMethod          Standard

pfset Wells.new_well.alltime.Extraction.Flux.water.Value        	     5.0
pfset Wells.new_well.alltime.Injection.Flux.water.Value		     7.5
pfset Wells.new_well.alltime.Injection.Concentration.water.tce.Fraction 0.1

\end{verbatim}\end{display}

\end{enumerate}
For more information on defining the problem,
see \S~\ref{Defining the Problem}.

%==============================
\subsection*{Converting the Pressure Output}

%% = we should delete or modify this since we don't use XPFTools anymore, although it still works.

Next, let us convert the pressure output computed by \parflow{}
to hydraulic head.
Do the following:
\begin{enumerate}

\item
Start the \file{xpftools} utility.
This is a utility which is used to do file conversions and other simple
operations on \parflow{} output files.

\item
To read in the data, use the \menu{Data} pulldown menu and select the
\menu{Load} \menu{ParFlow Binary} option.  

\item
In the file browser select the \file{default_single.out.press.00000.pfb}
file.

\item

Select the \menu{Compute Hydraulic Head} option from the \menu{Functions} 
pulldown menu.

\item 

In the \menu{Compute Hydraulic Head} popup, us the selector error next to
the \menu{Pressure Head} field to select \file{dataset0} (the
\file{default_single.out.press.00000.pfb} file).

\item 

Press the \menu{Compute Function} button at the bottom of the panel.  This
creates a new dataset labeled \file{dataset1}. 

\item
To save the hydraulic head data in a \parflow{} file,
\menu{Close} the \menu{Hydraulic Head} panel and select 
\menu{Save ParFlow Data}
from the \menu{Dataset} pulldown menu.

\item

In the file selector box type the filename
\file{default_single.out.hhead.00000.pfb} 
and save the dataset.  

You have now computed and created a new \parflow{} binary file with the
hydraulic head.

\end{enumerate}
For more information on manipulating and analyzing \parflow{} data,
see \S~\ref{Manipulating Data}.

%==============================
\section{ParFlow Solvers}
\label{ParFlow Solvers}

\parflow{} can operate using a number of different solvers.  Two of these solvers, IMPES (running in single-phase, fully-saturated mode, not multiphase) and RICHARDS are detailed below.  This is a brief summary of solver settings used to simulate under three sets of conditions, fully-saturated, variably-saturated and variably-saturated with overland flow.  A complete, detailed explanation of the solver parameters for ParFlow may be found later in this manual.
To simulate fully saturated, steady-state conditions set the solver to IMPES, an example is given below.  This is also the default solver in ParFlow, so if no solver is specified the code solves using IMPES.

\begin{verbatim}
pfset Solver               Impes
\end{verbatim}

To simulate variably-saturated, transient conditions, using Richards� equation, variably/fully saturated, transient w/ compressible storage set the solver to RICHARDS.  An example is below.  This is also the solver used to simulate surface flow or coupled surface-subsurface flow.

\begin{verbatim}
pfset Solver             Richards
\end{verbatim}

To simulate overland flow, using the kinematic wave approximation to the shallow-wave equations, set the solver to RICHARDS and set the upper patch boundary condition for the domain geometry to OverlandFlow, an example is below.  This simulates overland flow, independently or coupled to Richards� Equation as detailed in \cite{KM06}.  The overland flow boundary condition can simulate both uniform and spatially-distributed sources, reading a distribution of fluxes from a binary file in the latter case.  The two cases are set in a ParFlow input script as follows:

\begin{verbatim}
pfset Patch.z-upper.BCPressure.Type	OverlandFlow
\end{verbatim}
or
\begin{verbatim}
pfset Patch.z-upper.BCPressure.Type	OverlandFlowPFB
\end{verbatim}

For either case, the solver needs to be set to RICHARDS: 
\begin{verbatim}
pfset Solver		Richards
\end{verbatim}
and the jacobian is approximated:
\begin{verbatim}
pfset Solver.Nonlinear.UseJacobian		False
\end{verbatim}

In both cases the boundary fluxes may be set as a function of time cycle.  For the OverlandFlowPFB case:
\begin{verbatim}
pfset Patch.z-upper.BCPressure.Cycle		"rainrec"
pfset Patch.z-upper.BCPressure.rain.FileName	"bc.flux.test.1.pfb"
pfset Patch.z-upper.BCPressure.rec.FileName	"bc.flux.test.0.pfb"
\end{verbatim}

\parflow{} may also be coupled with the land surface model \code{CLM} \cite{Dai03}.  This version of \code{CLM} has been extensively modified to be called from within \parflow{} as a subroutine, to support parallel infrastructure including I/O and most importantly with modified physics to support coupled operation to best utilize the more sophisticated physics in \parflow{} \cite{MM05, KM08a}.  To couple \code{CLM} into \parflow{} first the \code{--with-clm} option is needed in the \code{./configure} command as indicated in \S~\ref{Installing ParFlow}. Second, the \code{CLM} module needs to be called from within \parflow{}, this is done using the following solver key:
\begin{verbatim}
pfset Solver.LSM CLM
\end{verbatim}
Note that this key is used to call \code{CLM} from within the nonlinear solver time loop and requires that the solver bet set to RICHARDS to work.  Note also that this key defaults to \emph{not} call \code{CLM} so if this line is omitted \code{CLM} will not be called from within \parflow{} even if compiled and linked.  Currently, \code{CLM} gets some of it's information from \parflow{} such as grid, topology and discretization, but also has some of it's own input files for land cover, land cover types and atmospheric forcing.