%=============================================================================
%=============================================================================

\chapter{The ParFlow System}
\label{The ParFlow System}

The \parflow{} system is still evolving, but at present,
it has three basic components.  GMS is used to create geometry input,
\parflow{} is the simulator itself, and AVS (or AVS/Express) are used
to visualize the results.  We discuss how to define the problem in
\S~\ref{Defining the Problem}, how to run \parflow{} in
\S~\ref{Running ParFlow}, and how to visualize the results in
\S~\ref{Visualizing Output with AVS}.  There is also a utility providing a set
of functions for manipulating \parflow{} data.  This utility is discussed in 
\S~\ref{Manipulating Data}.  Lastly, \S~\ref{Test Directory} discusses the
contents of a directory of test problems provided with \parflow{}.

%=============================================================================
%=============================================================================

\section{Defining the Problem}
\label{Defining the Problem}

Defining the problem may involve several steps.
One of these steps requires the use of
GMS (Groundwater Modeling System) \cite{GMS94tut,GMS95ref}
to define complicated geometries such as hydrostratigraphic layers.
These geometries are then converted to the
\file{.pfsol} file format (\S~\ref{ParFlow Solid Files (.pfsol)})
using the appropriate \pftools{} conversion utility
(\S~\ref{Manipulating Data}).

The ``main'' \parflow{} input file is the \file{.pftcl} file.
This input file is a TCL script with some special routines to
create a database which is used as the input for \parflow{}.
See \S~\ref{Main Input File (.pftcl)} for details on the format
of this file.
The input values into \parflow{} are defined by a key/value pair.  For
each key you provide the associated value using the \code{pfset} command
inside the input script.  To set the computational grid for the problem
you would enter:

\begin{display}\begin{verbatim}
#-----------------------------------------------------------------------------
# Computational Grid
#-----------------------------------------------------------------------------
pfset ComputationalGrid.Lower.X                -10.0
pfset ComputationalGrid.Lower.Y                 10.0
pfset ComputationalGrid.Lower.Z                  1.0
 
pfset ComputationalGrid.DX                       8.8888888888888893
pfset ComputationalGrid.DY                      10.666666666666666
pfset ComputationalGrid.DZ                       1.0
 
pfset ComputationalGrid.NX                      18
pfset ComputationalGrid.NY                      15
pfset ComputationalGrid.NZ                       8

\end{verbatim}\end{display}

The value is normally a single string, double, or integer.  In some cases,
in particular for a list of names, you need to supply a space seperated
sequence.  This can be done using either a double quote or bracies.

\begin{display}\begin{verbatim}
pfset Geom.domain.Patches "left right front back bottom top"

pfset Geom.domain.Patches {left right front back bottom top}
\end{verbatim}\end{display}

For commands longer than a single line, the TCL continuation character can be
used, 
\begin{display}\begin{verbatim}
pfset Geom.domain.Patches "very_long_name_1 very_long_name_2 very_long_name_3 \
                           very_long_name_4 very_long_name_5 very_long_name_6"
\end{verbatim}\end{display}


Since the input file is a TCL script you can use any feature of TCL to
define the problem.  This manual will make no effort to teach TCL so
refer to one of the available TCL manuals for more information
(``Practical Programming in TCL and TK'' by Brent Welch \cite{welch.95} 
is a good starting point).  
This is NOT required, you can get along fine without understanding TCL/TK.

Looking at the example programs in the \file{test} directory is one of
the best ways to understand what a \parflow{} input file looks like. See
\S~\ref{Test Directory}.  

%=============================================================================
%=============================================================================

\section{Running ParFlow}
\label{Running ParFlow}

Once the problem input is defined, you need to add a few things to 
the script to make it execute \parflow{}.  First you need to add
the TCL commands to load the \parflow{} command package.

\begin{display}\begin{verbatim}
#
# Import the ParFlow TCL package
#
lappend auto_path $env(PARFLOW_DIR)/bin 
package require parflow
namespace import Parflow::*
\end{verbatim}\end{display}

This loads the \code{pfset} and other \parflow{} commands into the
TCL shell.

Since this is a script you need to actually run \parflow{}. These are
normally the last lines of the input script.

\begin{display}\begin{verbatim}
#-----------------------------------------------------------------------------
# Run and Unload the ParFlow output files
#-----------------------------------------------------------------------------
pfrun default_single
pfundist default_single
\end{verbatim}\end{display}


The \code{pfrun} command runs \parflow{} with the database as it
exists at that point in the file.  The argument is the name to give to
the output files (which will normally be the same as the name of the
script).  Advanced users can set up multiple problems within the input
script by using different output names.  

The \code{pfundist} command takes the output files from the \parflow{}
run and undistributes them.  \parflow{} uses a virtual file system
which allows files to be distributed across the processors.  The
\code{pfundist} takes these files and collapses them into a single
file.  On some machines if you don't do the \code{pfundist} you will
see many files after the run.  Each of these contains the output from
a single node; before attempting using them you should undistribute them.

Since the input file is a TCL script run it using TCL:

\begin{display}\begin{verbatim}
tclsh runname.pftcl
\end{verbatim}\end{display}

NOTE: Make sure you are using TCL 8.0 or later.  The script will not
work with earlier releases.

For a network of workstations, you need to specify the computer
systems that are going to be used in the virtual machine (VM).  To add
a machine to the VM you should set the \code{Process.HostNames}
variable to be a space separated list of the machine names.

\begin{display}\begin{verbatim}
pfset Process.Hostnames ``bert.llnl.gov ernie.llnl.gov bigbird.llnl.gov''
\end{verbatim}\end{display}

One output file of particular interest is the \file{<run
name>.out.log} file.  This file contains information about the run
such as number of processes used, convergence history of algorithms,
timings and MFLOP rates.

\section{Restarting a Run}
\label{Restarting a Run}

Occasionally a \parflow{} run may need to be restarted because either a
system time limit has been reached and \parflow{} has been prematurely
terminated or the user specifically sets up a problem to run in segments.
In order to restart a run the user needs to know the conditions under which
\parflow{} stopped.  If \parflow{} was prematurely terminated then the
user must examine the output files from the last ``timed dump'' to see if
they are complete.  If not then those data files should be discarded and the
output files from the next to last ``timed dump'' will be used in the
restarting procedure.  As an important note, if any set of ``timed dump''
files are deleted remember to also delete corresponding lines in the well
output file or recombining the well output files from the individual segments
afterwards will be difficult.  It is not necessary to delete lines from
the log file as you will only be noting information from it.  To summarize,
make sure all the important output data files are complete, accurate and
consistent with each other.

Given a set of complete, consistent output files - to restart a run follow
this procedure :

\begin{enumerate}
   \item Note the important information for restarting :
   \begin{itemize}
      \item Write down the dump sequence number for the last collection of
           ``timed dump'' data.
      \item Examine the log file to find out what real time that ``timed dump''
            data was written out at and write it down.
   \end{itemize}
   \item Prepare input data files from output data files :
   \begin{itemize}
      \item Take all the concentration output files with the sequence number
            from above and convert them from a ParFlow Scattered Binary File
            \S~\ref{ParFlow Scattered Binary Files (.pfsb)} to a ParFlow
            Binary File \S~\ref{ParFlow Binary Files (.pfb)} (including a
            ``.dist'' file) using \pftools{} \S~\ref{Manipulating Data} and the
            {\it pfdist} utility in the input script.
   \end{itemize}
   \item Change the Main Input File \S~\ref{Main Input File (.pftcl)} :
   \begin{itemize}
      \item Edit the .pftcl file (you may want to save the old one) and
            utilize the read concentration input file option to specify
            all of the input files you created above as initial conditions
            for concentrations.
   \end{itemize}
   \item Restart the run :
   \begin{itemize}
      \item Utilizing an editor recreate all the input parameters used
            in the run except for the following two items :
            \begin{itemize}
               \item Use the dump sequence number from step 1
                     as the start\_count.
               \item Use the real time that the dump occured at from step 1
                     as the start\_time.
            \end{itemize}
   \end{itemize}
\end{enumerate}

%=============================================================================
%=============================================================================

\section{Visualizing Output with AVS}
\label{Visualizing Output with AVS}

We currently use AVS (Application Visualization System)
\cite{AVS92dguide,AVS92uguide,AVS93ref} to visualize \parflow{}
output.  AVS modules and networks are provided in directory
\file{\$PARFLOW_DIR/avs} for this purpose.  To access these modules
and networks within AVS, type
\begin{display}\begin{verbatim}
pfavs &
\end{verbatim}\end{display}
This will start AVS, load the \parflow{} module library, and set the default
network directory to your \file{\$PARFLOW_DIR/avs/networks} directory.
Click on \kbd{Network Editor} to open the network editor.
Here, you will be able to load supplied \parflow{} networks
(click on \kbd{Read Network}) and use \parflow{} modules to
either construct new networks or modify old networks
These provided networks may be used to visualize \parflow{} output.
Documentation for \parflow{} modules is found online using the
help capability provided with AVS.
See the AVS documentation for details on how to use AVS.

%=============================================================================
%=============================================================================

\section{Visualizing Output with AVS/Express}
\label{Visualizing Output with AVS/Express}

\subsection{Running and Interacting With AVS/Express}

Before running AVS/Express you may need to set some environment variables.
See the AVS/Express documentation for more information.

A script is included for starting AVS/Express for the local Express
installation.  This can be used as a template for other systems.  To
start Express with the \parflow{} modules loaded:

\begin{display}\begin{verbatim}
pfexpress
\end{verbatim}\end{display}

\subsection{Using the AVS/Express Applications}

First, some notes that apply to all of the applications.  A quick way to
reload the file currently being viewed is to press return in the text area
that contains the current file name.  To print the contents of a viewer
window (or save it to a \file{.ps} or \file{.eps} file), use the \menu{Print}
editor in the \menu{Editors} menu of the viewer.  To get to other user
interfaces, such as those for the grid scaling and downsizing, use the module
stack pulldown menu on the left side of the screen.

\begin{description}

\item[Animate]
    In the file dialog brought up by the ``Select file in sequence...'' button,
    the user should select one of the \file{.pfsb} files from the sequence
    that will be animated.  The application will then load the first file in
    the sequence and do the isosurface on it.  The desired isosurface level
    should be selected at this point also.  Start, end, and stride are used
    to select the files to include in the animation.  The indices in these
    fields are from 0 to the number of \file{.pfsb} files in the series minus
    one.  If cycle is checked, the animation will continuously cycle while
    run is checked.  Checking reset will cause the display to be reset to the
    start file.  Animate only works with \file{.pfsb} files.

\item[Brick]
    Everything here should be self-explanatory.  The dials or sliders control
    the upper bounding planes of the box.  The dials have the immediate
    toggle set, causing them to update the display as they are being
    modified.  The sliders do not have this toggle set.

\item[ExcavateBrick]
    The ``opposite'' of Brick, but with the same user interface.
 
\item[Isosurface]
    Like Animate, but without the animation stuff and only allows you to view
    one file at a time.  Also allows you to view \file{.pfb} files.
 
\item[Orthograph]
    Axis labels and names are now working, although they tend to get jumbled
    together at the origin.  The Probe Value field in the user interface
    gives the value of the current probe location in the plot.  You can probe
    the value of an isoline by control-left mouse button clicking anywhere on
    it.  You can also choose up to two isolines at specific locations by
    modifying the sliders in the isoline user interface in the module stack.
    If more precision is necessary for specifying these isolines values, a
    new user interface can easily be added for it.
 
    One note on a semi-bug: If the plot shrinks down to become very small,
    try hitting the Normalize button.  If that doesn't get it back to normal,
    flip between different axes a time or two, and hopefully that should
    work.  As a last-ditch effort, delete and reload the application.  This
    problem seems to be rare, but I have always been able to solve it without
    having to reload the application.

\item[Slice3]
    The same user interface as Brick.

\end{description}

Of these applications, there is currently one known bug.  In Isosurface,
loading a \file{.pfb} file after a \file{.pfsb} file causes minimum values in
some places to become messed up, but the application is still usable.  The is
due to an unfortunate ``feature'' in the AVS/Express UIdial module.  To avoid
this, delete and reload the application when you want to visualize a
\file{.pfb} file after a \file{.pfsb} file.

%=============================================================================
%=============================================================================

\section{Manipulating Data}
\label{Manipulating Data}

\subsection{Introduction to the \parflow{} TCL commands (PFTCL) }

Several tools for manipulating data are provided in PFTCL command set.
In order to use them you need to load the \parflow{} package into
the TCL shell.  If you are doing simple data manipulation the
\file{xpftools} provides GUI access to most of these features.
All of these tools are accessible inside of a \parflow{} input script.
You can use them to do post and pre processing of datafiles each time
you execute a run.

\begin{display}\begin{verbatim}
#
# To Import the ParFlow TCL package
#
lappend auto_path $env(PARFLOW_DIR)/bin 
package require parflow
namespace import Parflow::*
\end{verbatim}\end{display}

Use \kbd{pfhelp} to get a list of commands.

PFTCL assigns identifiers to each data set it stores.
For example, if you read in a file called \file{foo.pfb},
you get the following:
\begin{display}\begin{verbatim}
parflow> pfload foo.pfb
dataset0
\end{verbatim}\end{display}
The first line is typed in by the user and the second line
is printed out by PFTCL.
It indicates that the data read from file \file{foo.pfb} is
associated with the identifier \code{dataset0}.

To exit use the standard TCL command \kbd{exit}.

\subsection{PFTCL Commands}
\label{PFTCL Commands}
The following gives a list of \parflow{} commands and instructions for their use:
Note that commands that perform operations on data sets will
require an identifier for each data set it takes as input.

\begin{description}

\item{\begin{verbatim}pfaxpy alpha x y\end{verbatim}}
This command computes y = alpha*x+y where alpha is a scalar and x and 
y are identifiers representing data sets.  No data set identifier is
returned upon successful completion since data set y is overwritten.

\item{\begin{verbatim}pfcvel conductivity phead\end{verbatim}}
This command computes the Darcy velocity in cells for the conductivity data set
represented by the identifier `conductivity' and the pressure head
data set represented by the identifier `phead'.  (note: This "cell"
is not the same as the grid cells; its corners are defined by the
grid vertices.)  The identifier of the data set created by this
operation is returned upon successful completion.

\item{\begin{verbatim}pfdelete dataset\end{verbatim}}
This command deletes the data set represented by the identifier `dataset'.
        
\item{\begin{verbatim}pfdiffelt datasetp datasetq i j k digits [zero]\end{verbatim}}
This command returns the difference of two corresponding coordinates
from `datasetp' and `datasetq' if the number of digits in agreement
(significant digits) differs by more than `digits' significant
digits and the difference is greater than the absolute zero given     
by `zero'.


\item{\begin{verbatim}pfdist filename \end{verbatim}}
Distribute the file onto the virtual file system.  This utility must
be used to create files which \parflow{} can use as input.  \parflow{}
uses a virtual file system which allows each node of the parallel
machine to read from the input file independentaly.  The utility does
the inverse of the pfundist command.  If you are using a \parflow{}
binary file for input you should do a pfdist just before you do the
pfrun.  This command requires that the processor topology and
computational grid be set in the input file so that it knows how to
distribute the data.

\item{\begin{verbatim}pfflux conductivity hhead\end{verbatim}}
This command computes the net Darcy flux at vertices for the
conductivity data set `conductivity' and the hydraulic head data      
set given by `hhead'.  An identifier representing the flux computed   
will be returned upon successful completion.
        
\item{\begin{verbatim}pfgetelt dataset i j k\end{verbatim}}
This command returns the value at element (i,j,k) in data set         
`dataset'.  The i, j, and k above must range from 0 to (nx - 1), 0 to 
(ny - 1), and 0 to (nz - 1) respectively.  The values nx, ny, and nz
are the number of grid points along the x, y, and z axes respectively.
The string `dataset' is an identifier representing the data set whose
element is to be retrieved.
        
\item{\begin{verbatim}pfgetgrid dataset\end{verbatim}}
This command returns a description of the grid which serves as the
domain of data set `dataset'.  The format of the description is given 
below.
\begin{itemize}
\item{\begin{verbatim}(nx, ny, nz)\end{verbatim}
The number of coordinates in each direction.}
\item{\begin{verbatim}(x, y, z)\end{verbatim}The origin of the grid.}
\item{\begin{verbatim}(dx, dy, dz)\end{verbatim}The distance between each
coordinate in each direction.}
\end{itemize}
The above information is returned in the following Tcl list format:
{nx ny nz} {x y z} {dx dy dz}
        
\item{\begin{verbatim}pfgridtype gridtype\end{verbatim}}
This command sets the grid type to either cell centered if `gridtype'
is set to `cell' or vetex centered if `gridtype' is set to `vertex'.
If no new value for `gridtype' is given, then the current value of
`gridtype' is returned.  The value of `gridtype' will be returned upon
successful completion of this command.

\item{\begin{verbatim}pfhhead phead\end{verbatim}}
This command computes the hydraulic head from the pressure head       
represented by the identifier `phead'.  An identifier for the 
hydraulic head computed is returned upon successful completion.

\item{\begin{verbatim}pflistdata dataset\end{verbatim}}
This command returns a list of pairs if no argument is given.  The
first item in each pair will be an identifier representing the data   
set and the second item will be that data set's label.  If a data     
set's identifier is given as an argument, then just that data set's   
name and label will be returned.
       
\item{\begin{verbatim}pfload [file format] filename\end{verbatim}}

Loads a dataset into memory so it can be manipulated using the other
utilities.  A file format may preceed the filename in order to
indicate the file's format.  If no file type option is given, then the
extension of the filename is used to determine the default file type.
An identifier used to represent the data set will be returned upon
successful completion.

      File type options include:
\begin{itemize}
\item{\begin{verbatim}pfb\end{verbatim}} ParFlow binary format.  
Default file type for files with a `.pfb' extension.
\item{\begin{verbatim}pfsb\end{verbatim}}  ParFlow scattered binary format.
Default file type for files with a `.pfsb' extension.
\item{\begin{verbatim}sa\end{verbatim}}  ParFlow simple ASCII format.
Default file type for files with a `.sa' extension.
\item{\begin{verbatim}sb\end{verbatim}} ParFlow simple binary format.
Default file type for files with a `.sb' extension.
\item{\begin{verbatim}rsa\end{verbatim}} ParFlow real scattered ASCII format.
Default file type for files with a `.rsa' extension
\end{itemize}
      
\item{\begin{verbatim}pfloadsds filename dsnum\end{verbatim}}
This command is used to load Scientific Data Sets from HDF files.    
The SDS number `dsnum' will be used to find the SDS you wish to load
from the HDF file `filename'.  The data set loaded into memory will
be assigned an identifier which will be used to refer to the data set
until it is deleted.  This identifier will be returned upon
successful completion of the command.  
        
\item{\begin{verbatim}pfmdiff datasetp datasetq digits [zero]\end{verbatim}}
If `digits' is greater than or equal to zero, then this command     
computes the grid point at which the number of digits in agreement    
(significant digits) is fewest and differs by more than `digits'    
significant digits.  If `digits' is less than zero, then the point  
at which the number of digits in agreement (significant digits) is    
minimum is computed.  Finally, the maximum absolute difference is     
computed.  The above information is returned in a Tcl list
of the following form:
{mi mj mk sd} adiff
   
Given the search criteria, (mi, mj, mk) is the coordinate where the
minimum number of significant digits `sd' was found and `adiff' is
the maximum absolute difference.        
        
        
\item{\begin{verbatim}pfnewdata {nx ny nz} {x y z} {dx dy dz} label\end{verbatim}}
This command creates a new data set whose dimension is described by
the lists {nx ny nz}, {x y z}, and {dx dy dz}.  The first list,
describes the dimensions, the second indicates the origin, and the
third gives the length intervals between each coordinate along each
axis.  The `label' argument will be the label of the data set that
gets created.  This new data set that is created will have all of
its data points set to zero automatically.  An identifier for the new
data set will be returned upon successful completion.


\item{\begin{verbatim}pfnewlabel dataset newlabel\end{verbatim}}
This command changes the label of the data set `dataset' to
`newlabel'.
 
\item{\begin{verbatim}pfphead hhead\end{verbatim}}
This command computes the pressure head from the hydraulic head   
represented by the identifier `hhead'.  An identifier for the pressure
head is returned upon successful completion.
        
\item{\begin{verbatim}pfsavediff datasetp datasetq digits [zero] -file filename
\end{verbatim}}
This command saves to a file the differences between the values
of the data sets represented by `datasetp' and `datasetq' to file
`filename'.  The data points whose values differ in more than         
`digits' significant digits and whose differences are greater than  
`zero' will be saved.  Also, given the above criteria, the
minimum number of digits in agreement (significant digits) will be    
saved.

If `digits' is less than zero, then only the minimum number of
significant digits and the coordinate where the minimum was
computed will be saved.

In each of the above cases, the maximum absolute difference given
the criteria will also be saved.

\item{\begin{verbatim}pfsave dataset -filetype filename\end{verbatim}}
This command is used to save the data set given by the identifier
`dataset' to a file `filename' of type `filetype' in one of the
ParFlow formats below.

File type options include:
\begin{itemize}
\item{pfb}  ParFlow binary format.
\item{sa}  ParFlow simple ASCII format.
\item{sb}  ParFlow simple binary format.
\item{vis}  Vizamrai binary format.
\end{itemize}

\item{\begin{verbatim}pfsavesds dataset -filetype filename\end{verbatim}}
This command is used to save the data set represented by the
identifier `dataset' to the file `filename' in the format given by    
`filetype'.  

The possible HDF formats are:
\begin{itemize}
\item{-float32}
\item{-float64}
\item{-int8}
\item{-uint8}
\item{-int16}
\item{-uint16}
\item{-int32}
\item{-uint32}
\end{itemize}

       
\item{\begin{verbatim}pfstats dataset\end{verbatim}}
This command prints various statistics for the data set represented by
the identifier `dataset'.  The minimum, maximum, mean, sum, variance,
and standard deviation are all computed.  The above values are
returned in a list of the following form:
{min max mean sum variance (standard deviation)}
        
        
\item{\begin{verbatim}pfvmag datasetx datasety datasetz\end{verbatim}}
This command computes the velocity magnitude when given three velocity
components.  The three parameters are identifiers which represent
the x, y, and z components respectively.  The identifier of the data
set created by this operation is returned upon successful completion.


\item{\begin{verbatim}pfvvel conductivity phead\end{verbatim}}
This command computes the Darcy velocity in cells for the conductivity
data set represented by the identifier `conductivity' and the pressure
head data set represented by the identifier `phead'.  The identifier  
of the data set created by this operation is returned upon successful 
completion.  

        
\item{\begin{verbatim}pfprintdata dataset\end{verbatim}}
This command executes `pfgetgrid' and `pfgetelt' in order to display
all the elements in the data set represented by the identifier
`dataset'.
        
        
\item{\begin{verbatim}pfprintdiff datasetp datasetq digits [zero]\end{verbatim}}
This command executes `pfdiffelt' and `pfmdiff' to print differences
to standard output.  The differences are printed one per line along
with the coordinates where they occur.  The last two lines displayed  
will show the point at which there is a minimum number of significant 
digits in the difference as well as the maximum absolute difference.
        
        
\item{\begin{verbatim}pfprintgrid dataset\end{verbatim}}
This command executes pfgetgrid and formats its output before printing
it on the screen.  The triples (nx, ny, nz), (x, y, z), and
(dx, dy, dz) are all printed on seperate lines along with labels
describing each.
        
        
\item{\begin{verbatim}pfprintlist [dataset]\end{verbatim}}
This command executes pflistdata and formats the output of that
command.  The formatted output is then printed on the screen.  The
output consists of a list of data sets and their labels one per line
if no argument was given or just one data set if an identifier was
given.
        
        
\item{\begin{verbatim}pfprintmdiff datasetp datasetq digits [zero]\end{verbatim}}
This command executes `pfmdiff' and formats that command's output
before displaying it on the screen.  Given the search criteria, a line
displaying the point at which the difference has the least number of
significant digits will be displayed.  Another line displaying the
maximum absolute difference will also be displayed.
 
        
\item{\begin{verbatim}printstats dataset\end{verbatim}}
This command executes `pfstats' and formats that command's output
before printing it on the screen.  Each of the values mentioned in the
description of `pfstats' will be displayed along with a label.


\item{\begin{verbatim}pfundist filename, pfundist runname\end{verbatim}}

The command undistributes a \parflow{} output file.  \parflow{} uses a
distributed file system where each node can write to its own file.
The pfundist command takes all of these individual files and collapses
them into a single file.

The arguments can be a runname or a filename.  If a runname is given
then all of the output files associated with that run are
undistributed.

Normally this is done after every pfrun command.
        
\end{description}        

%=============================================================================
%=============================================================================

\section{Directory of Test Cases}
\label{Test Directory}

\parflow{} comes with a directory containing a few simple input files for use
as templates in making new files and for use in testing the code.  This section
gives a brief descriptionn of the problems in this driectory.
