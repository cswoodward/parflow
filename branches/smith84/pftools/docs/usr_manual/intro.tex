%=============================================================================
% Chapter: Introduction
%=============================================================================

\chapter{Introduction}
\label{Introduction}

\parflow{} is a groundwater flow and contaminant transport simulation code
designed to run efficiently in a multi-processor computing environment
on a variety of available machines.  This simulator is being built by
scientists from the Center for Applied Scientific Computing (CASC),
Environmental Programs, and the Environmental Protection Department.
\parflow{} is able to utilize the computing power that is
available on today's supercomputers, enabling it to more accurately model
the effects of subsurface heterogeneity on fluid transport processes.
Heterogeneity is present on many scales in the subsurface, ranging
from the microscopic capillary processes to macroscopic variations in
structure, such as fractures and faults.
\parflow{} attempts to account for heterogeneity and model its
effects on contaminant flow.

This manual describes how to use \parflow{}, and is intended for
geoscientists and environmental scientists and engineers.  In
Chapter~\ref{Getting Started}, we describe how to install \parflow{}.
Then, we lead the user through a simple \parflow{} run.  In
Chapter~\ref{The ParFlow System}, we describe the \parflow{} system in
more detail.  Chapter~\ref{ParFlow Files} describes the formats of the
various files used by \parflow{}.

