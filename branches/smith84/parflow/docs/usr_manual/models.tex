%=============================================================================
% Chapter: Model Equations
%=============================================================================

\chapter{Model Equations}
\label{Model_Equations}

In this chapter, we discuss the model equations used by \parflow{} for both its
multiphase flow and transport model and the variably saturated flow model.
In section \ref{Multi-Phase Flow Equations} we describe the multi-phase
flow equations, and in section \ref{Transport Equations} we describe
the transport equations.  Lastly, section \ref{Richards' Equation} 
describes the Richards' equation model for variably saturated flow as 
implemented in \parflow{}.  

\section{Multi-Phase Flow Equations}
\label{Multi-Phase Flow Equations}

The flow equations are a set of {\em mass balance} and
{\em momentum balance} (Darcy's Law) equations, given respectively by,
\begin{equation} \label{eqn-mass-balance}
\frac{\partial}{\partial t} ( \Xpor \XSi )
  ~+~ \Xgrad \cdot \XVi
  ~-~ \XQi ~=~ 0 ,
\end{equation}
\begin{equation} \label{eqn-darcy}
\XVi ~+~ \Xmobi \cdot ( \Xgrad \Xpi ~-~ \Xdeni \Xg ) ~=~ 0 ,
\end{equation}
for $i = 0, \ldots , \Xnphases - 1$ $(\Xnphases \in \{1,2,3\})$, where
\begin{eqnarray} \label{eqn-phase-mobility}
\Xmobi & = & \frac{\Xk \Xkri}{\Xvisi} , \\
\Xg    & = & [ 0, 0, -g ]^T ,
\end{eqnarray}
Table \ref{table-flow-units} defines the symbols in the above equations,
and outlines the symbol dependencies and units.
\begin{table} \center
\caption{Notation and units for flow equations.}
\smallskip
\begin{tabular}{||c||c|c||}
\hline
symbol & quantity & units \\
\hline\hline
$\Xpor (\Xx,t)$ & porosity                      & []                  \\ \hline
$\XSi (\Xx,t)$  & saturation                    & []                  \\ \hline
$\XVi (\Xx,t)$  & Darcy velocity vector         & [$L T^{-1}$]        \\ \hline
$\XQi (\Xx,t)$  & source/sink                   & [$T^{-1}$]          \\ \hline
$\Xmobi$        & mobility                      & [$L^{3} T M^{-1}$]  \\ \hline
$\Xpi (\Xx,t)$  & pressure                      & [$M L^{-1} T^{-2}$] \\ \hline
$\Xdeni$        & mass density                  & [$M L^{-3}$]        \\ \hline
$\Xg$           & gravity vector                & [$L T^{-2}$]        \\ \hline
$\Xk (\Xx,t)$   & intrinsic permeability tensor & [$L^{2}$]           \\ \hline
$\Xkri (\Xx,t)$ & relative permeability         & []                  \\ \hline
$\Xvisi$        & viscosity                     & [$M L^{-1} T^{-1}$] \\ \hline
$g$             & gravitational acceleration    & [$L T^{-2}$]        \\ \hline
\end{tabular}
\label{table-flow-units}
\end{table}
Here, $\Xpor$ describes the fluid capacity of the porous medium,
and $\XSi$ describes the content of phase $i$ in the porous medium,
where we have that $0 \le \Xpor \le 1$ and $0 \le \XSi \le 1$.
The coefficient $\Xk$ is considered a scalar here.
We also assume that $\Xdeni$ and $\Xvisi$ are constant.
Also note that in \parflow{}, we assume that the relative permeability
is given as $\Xkri (\XSi)$.
The Darcy velocity vector is related to the {\em velocity vector}, $\Xvi$,
by the following:
\begin{equation} \label{eqn-Dvec-vs-vvec}
\XVi = \Xpor \XSi \Xvi .
\end{equation}

To complete the formulation, we have the following
$\Xnphases$ {\em consititutive relations}
\begin{equation} \label{eqn-constitutive-sum}
\sum_i \XSi = 1 ,
\end{equation}
\begin{equation} \label{eqn-constitutive-capillary}
\Xp_{i0} ~=~ \Xp_{i0} ( \XS_0 ) ,
~~~~~~ i = 1 , \ldots , \Xnphases - 1 .
\end{equation}
where, $\Xp_{ij} = \Xp_i - \Xp_j$ is the {\em capillary pressure} between
phase $i$ and phase $j$.
We now have the $3 \Xnphases$ equations,
(\ref{eqn-mass-balance}), (\ref{eqn-darcy}),
(\ref{eqn-constitutive-sum}), and (\ref{eqn-constitutive-capillary}),
in the $3 \Xnphases$ unknowns,
$\XSi, \XVi$, and $\Xpi$.

For technical reasons, we want to rewrite the above equations.
First, we define the {\em total mobility}, $\XmobT$,
and the {\em total velocity}, $\XVT$, by the relations
\begin{eqnarray}
\XmobT ~=~ \sum_{i} \Xmobi , \label{eqn-total-mob} \\
\XVT ~=~ \sum_{i} \XVi .     \label{eqn-total-vel}
\end{eqnarray}
After doing a bunch of algebra, we get the following equation
for $\Xp_0$:
\begin{equation} \label{eqn-pressure}
-~ \sum_{i}
  \left \{
    \Xgrad \cdot \Xmobi
      \left ( \Xgrad ( \Xp_0 ~+~ \Xp_{i0} ) ~-~ \Xdeni \Xg \right )
    ~+~
    \XQi
  \right \}
~=~ 0 .
\end{equation}
After doing some more algebra, we get the following $\Xnphases - 1$
equations for $\XSi$:
\begin{equation} \label{eqn-saturation}
\frac{\partial}{\partial t} ( \Xpor \XSi )
~+~
\Xgrad \cdot
  \left (
     \frac{\Xmobi}{\XmobT} \XVT ~+~
     \sum_{j \neq i} \frac{\Xmobi \Xmob_j}{\XmobT} ( \Xden_i - \Xden_j ) \Xg
  \right )
~+~
\sum_{j \neq i} \Xgrad \cdot
    \frac{\Xmobi \Xmob_j}{\XmobT} \Xgrad \Xp_{ji}
~-~ \XQi
~=~ 0 .
\end{equation}
The capillary pressures $\Xp_{ji}$ in (\ref{eqn-saturation}) are
rewritten in terms of the constitutive relations in
(\ref{eqn-constitutive-capillary}) so that we have
\begin{equation} \label{eqn-derived-capillary}
\Xp_{ji} ~=~ \Xp_{j0} ~-~ \Xp_{i0} ,
\end{equation}
where by definition, $\Xp_{ii} = 0$.
Note that equations (\ref{eqn-saturation}) are analytically the
same equations as in (\ref{eqn-mass-balance}).
The reason we rewrite them in this latter form is because
of the numerical scheme we are using.
We now have the $3 \Xnphases$ equations,
(\ref{eqn-pressure}), (\ref{eqn-saturation}),
(\ref{eqn-total-vel}), (\ref{eqn-darcy}), and
(\ref{eqn-constitutive-capillary}),
in the $3 \Xnphases$ unknowns,
$\XSi, \XVi$, and $\Xpi$.


\section{Transport Equations}
\label{Transport Equations}

The transport equations in \parflow{} are currently defined as follows:
\begin{eqnarray} \label{eqn-transport}
\left ( \frac{\partial}{\partial t} (\Xpor \Xcij) ~+~ \Xdegj ~ \Xpor \Xcij \right ) & + & \Xgrad \cdot \left ( \Xcij \XVi \right ) \nonumber \\
& = & \\
-\left ( \frac{\partial}{\partial t} ((1 - \Xpor) \Xsolidden \XFij) ~+~  \Xdegj ~ (1 - \Xpor) \Xsolidden \XFij \right ) & + & \sum_{k}^{\XnI} \XwellrateIik \chi_{\XindIk} \left ( \Xcij - \Xcbarkij \right ) ~-~ \sum_{k}^{\XnE} \XwellrateEik \chi_{\XindEk} \Xcij \nonumber
\end{eqnarray}
where $i = 0, \ldots , \Xnphases - 1$ $(\Xnphases \in \{1,2,3\})$
is the number of phases,
$j = 0, \ldots , \Xnc - 1$ is the number of contaminants, and where
$\Xcij$ is the concentration of contaminant $j$ in phase $i$.  Recall also,
that $\chi_A$ is the characteristic function of set $A$, i.e. $\chi_A(x) = 1$
if $x \in A$ and $\chi_A(x) = 0$ if $x \not\in A$.
Table \ref{table-transport-units} defines the symbols in the above equation,
and outlines the symbol dependencies and units.  The equation is basically
a statement of mass conservation in a convective flow (no diffusion) with
adsorption and degradation effects incorporated along with the addition of
injection and extraction wells.
\begin{table} \center
\caption{Notation and units for transport equation.}
\smallskip
\begin{tabular}{||c||c|c||}
\hline
symbol & quantity & units \\
\hline\hline
$\Xpor (\Xx)$       & porosity                        & []               \\ \hline
$\Xcij (\Xx,t)$     & concentration fraction          & []               \\ \hline
$\XVi (\Xx,t)$      & Darcy velocity vector           & [$L T^{-1}$]     \\ \hline
$\Xdegj$            & degradation rate                & [$T^{-1}$]       \\ \hline
$\Xsolidden (\Xx)$  & density of the solid mass       & [$M L^{-3}$]]    \\ \hline
$\XFij  (\Xx, t)$   & mass concentration              & [$L^{3} M^{-1}$] \\ \hline
$\XnI$              & number of injection wells       & []               \\ \hline
$\XwellrateIik (t)$ & injection rate                  & [$T^{-1}$]       \\ \hline
$\XindIk (\Xx)$     & injection well region           & []               \\ \hline
$\Xcbarkij ()$      & injected concentration fraction & []               \\ \hline
$\XnE$              & number of extraction wells      & []               \\ \hline
$\XwellrateEik (t)$ & extraction rate                 & [$T^{-1}$]       \\ \hline
$\XindEk (\Xx)$     & extraction well region          & []               \\ \hline
\end{tabular}
\label{table-transport-units}
\end{table}
These equations will soon have to be generalized to include a diffusion term.
At the present time, as an adsorption model, we take the mass concentration
term ($\XFij$) to be instantaneous in time and a linear function of contaminant
concentration :
\begin{equation} \label{eqn-linear-retardation}
\XFij = \XKdj \Xcij,
\end{equation}
where $\XKdj$ is the distribution coefficient of the component
([$L^{3} M^{-1}$]).
If \ref{eqn-linear-retardation} is substituted into \ref{eqn-transport}
the following equation results (which is the current model used in \parflow{}) :
\begin{eqnarray} \label{eqn-transport2}
(\Xpor + (1 - \Xpor) \Xsolidden \XKdj) \frac{\partial}{\partial t} \Xcij & + & \Xgrad \cdot \left ( \Xcij \XVi \right ) \nonumber \\
& = & \nonumber \\
-~(\Xpor + (1 - \Xpor) \Xsolidden \XKdj) \Xdegj \Xcij & + & \sum_{k}^{\XnI} \XwellrateIik \chi_{\XindIk} \left ( \Xcij - \Xcbarkij \right ) ~-~ \sum_{k}^{\XnE} \XwellrateEik \chi_{\XindEk} \Xcij
\end{eqnarray}

\section{Notation and Units}
\label{Notation and Units}

In this section, we discuss other common formulations of the flow
and transport equations, and how they relate to the equations solved
by \parflow{}.

We can rewrite equation (\ref{eqn-darcy}) as
\begin{equation} \label{eqn-darcy-b}
\XVi ~+~ \XKi \cdot ( \Xgrad \Xhi ~-~ \frac{\Xdeni}{\Xscale} \Xg ) ~=~ 0 ,
\end{equation}
where
\begin{eqnarray} \label{eqn-cond-phead}
\XKi & = & \Xscale \Xmobi , \\
\Xhi & = & ( \Xpi ~-~ \Xgp ) / \Xscale .
\end{eqnarray}
Table \ref{table-flow-units-b} defines the symbols and their units.
\begin{table} \center
\caption{Notation and units for reformulated flow equations.}
\smallskip
\begin{tabular}{||c||c|c||}
\hline
symbol & quantity & units \\
\hline\hline
$\XVi$      & Darcy velocity vector         & [$L T^{-1}$]        \\ \hline
$\XKi$      & hydraulic conductivity tensor & [$L T^{-1}$]        \\ \hline
$\Xhi$      & pressure head                 & [$L$]               \\ \hline
$\Xscale$   & constant scale factor         & [$M L^{-2} T^{-2}$] \\ \hline
$\Xg$       & gravity vector                & [$L T^{-2}$]        \\ \hline
\end{tabular}
\label{table-flow-units-b}
\end{table}
We can then rewrite equations (\ref{eqn-pressure}) and
(\ref{eqn-saturation}) as
\begin{equation} \label{eqn-pressure-b}
-~ \sum_{i}
  \left \{
    \Xgrad \cdot \XKi
      \left ( \Xgrad ( \Xh_0 ~+~ \Xh_{i0} ) ~-~
        \frac{\Xdeni}{\Xscale} \Xg \right )
    ~+~
    \XQi
  \right \}
~=~ 0 ,
\end{equation}
\begin{equation} \label{eqn-saturation-b}
\frac{\partial}{\partial t} ( \Xpor \XSi )
~+~
\Xgrad \cdot
  \left (
     \frac{\XKi}{\XKT} \XVT ~+~
     \sum_{j \neq i} \frac{\XKi \XK_j}{\XKT}
       \left ( \frac{\Xden_i}{\Xscale} - \frac{\Xden_j}{\Xscale} \right ) \Xg
  \right )
~+~
\sum_{j \neq i} \Xgrad \cdot
    \frac{\XKi \XK_j}{\XKT} \Xgrad \Xh_{ji}
~-~ \XQi
~=~ 0 .
\end{equation}

Note that $\XKi$ is supposed to be a tensor, but we treat it as
a scalar here.
Also, note that by carefully defining the input to \parflow{}, we can
use the units of equations (\ref{eqn-pressure-b}) and
(\ref{eqn-saturation-b}).
To be more precise, let us denote \parflow{} input symbols by appending
the symbols in table \ref{table-flow-units} with $(I)$, and
let $\Xscale = \Xden_0 g$ (this is a typical definition).
Then, we want:
\begin{eqnarray} \label{eqn-parflow-input}
\Xk (I)    & = & \Xscale \Xk / \Xvis_0 ; \\
\Xvisi (I) & = & \Xvisi / \Xvis_0 ; \\
\Xpi (I)   & = & \Xhi ; \\
\Xdeni (I) & = & \Xdeni / \Xden_0 ; \\
g (I)      & = & 1 .
\end{eqnarray}
By doing this, $\Xk (I)$ represents hydraulic conductivity of the base
phase $\XK_0$ (e.g. water) under saturated conditions (i.e. $k_{r0} = 1$).
Though \parflow{} input has been defined this way in the past, this is not
the recommended procedure since it may lead to confusion and mistakes.

\section{Richards' Equation}
\label{Richards' Equation}

The form of Richards' equation implemented in \parflow{} is given as,
\begin{eqnarray}
\frac{\partial (S(p)\rho(p)\phi)}{\partial t} 
- \dvop (K(p)\rho(p)(\nabla p - \rho(p) \Xg)) = Q, \;  {\rm in} \; \Omega, 
\label{eq:richard}
\end{eqnarray}
where $\Omega$ is the flow domain, $p$ is the water pressure, $S$ is the water
saturation, $\phi$ is the porosity of the medium, $K(p)$ is the hydraulic
conductivity  and $Q$ is the water source/sink term (includes wells).
The hydraulic conductivity can be written as,
\begin{eqnarray}
K(p) =  \frac{\Xk k_r(p)}{\mu}
\end{eqnarray}
Boundary conditions can be stated as,
\begin{eqnarray}
p & = & p_D, \; {\rm on} \; \Gamma^D, \label{eq:bcd} \\
-K(p)\nabla p \cdot {\bf n} & = & 
g_N, \; {\rm on} \; \Gamma^N, \label{eq:bcn}
\end{eqnarray}
where $\Gamma^D \cup \Gamma^N = \partial \Omega$, $\Gamma^D \neq \emptyset$,
and ${\bf n}$ is an outward pointing, unit, normal vector to $\Omega$.
This is the mixed form of Richards' equation. 
Note here that due to the constant (or passive) air phase pressure assumption,
Richards' equation ignores the air phase except through its
effects on the hydraulic conductivity, $K$.  
An initial condition,
\begin{eqnarray}
p = p^0(x), \; t = 0,
\end{eqnarray}
completes the specification of the problem.
